\documentclass[./dissertation.tex]{subfiles}
\begin{document}

    \chapter{Enter Title Here}
    This is where your content will go. There will be examples of figures, tables, citations, and footnotes/sidenotes in this section. In the source files, there will be examples and how to get them to appear on the TOC, LOF, and LOTs Some Diciplines use section and subsection headings more than others. The Chapter one is really all that matters. This is strictly to give the reader a sence of how they look on the TOC.
    \section{Your first section}
      How do you feel so far? In these next few subsections we are going to be illustrating how to do these different things in LaTeX. Hopefully you will be able to replicate it whenever you need them.
      It is good to have all of you images (figures) in their own folder so you can keep some organization to your projects.

      \subsection{Table Example}
        In this subsection, we are going to give an example of a table and how one will look. There are many ways to make a table and customize them. Here is one example:


      \begin{table}[ht]
        \centering
      \begin{tabular}[c]{|l|c|rc|}
        \hline
        left justified  & centerd  & right justified  & no left border\\
        \hline
        row 1 & fill & in between & the \&s\\
        row 2 & & &\\
        \hline

      \end{tabular}
      \caption{The Caption of the table.}
      \label{table:someTable}
      \end{table}



      \subsection{Figure Example}
      This is going to be an example of how to insert an image and it's caption. We can also reference it anywhere else in the document as well.
      \begin{figure}[h]
        \centering\includegraphics[width=0.5\textwidth]{figures/digilab_logo}
        \caption{The Digital Humanities Lab Logo}
        \label{fig:digilogo}
      \end{figure}
      The Digital Humanities logo, figure \ref{fig:digilogo} is one of many different prototypes.

      \subsection{Footnotes and Sidenotes}
      


      \subsection{Equation Examples}
        You can also reference an equation just like a figure and table. There are two different enviornments that are needed. The first is just using the equation enviornment. This is useful when you have one equation to write. As seen here:
          \begin{equation*}
            Y=\beta_{0} + \sum\limits_{i=1}^n \beta_{i}X_{i} + e
          \end{equation*}
        The equation for the general multiple linear regression model is above. This is a good use case for the equation enviornment. \LaTeX knows to expect math symbols inside the enviornment. What if you need more than one line? What if each line need to align? Thats when the align enviornment comes in handy. A lot of the times, you will need to show the simplification of equations or steps in calcualtions.
          \begin{center}
            \begin{align*}
              P_{s} &= \frac{D*F}{N*P*I}\\
                            \\
              &= \frac{D*R*W}{N*P*I} \text{  since } F=R*W\\
                      \\
              0.70 &= \frac{1*R*0.1}{2.5*1*4,294,967,295}\\
                      \\
             R &= 751619276625.0 \\
             \\
             A=\frac{R*0.1}{4096} &\implies A = 1835007.99
           \end{align*}
         \end{center}

\end{document}
